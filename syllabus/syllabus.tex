% Don't touch this %%%%%%%%%%%%%%%%%%%%%%%%%%%%%%%%%%%%%%%%%%%
\documentclass[11pt]{article}
\usepackage{fullpage}
\usepackage[left=1in,top=1in,right=1in,bottom=1in,headheight=3ex,headsep=3ex]{geometry}
\usepackage{graphicx}
\usepackage{float}

\newcommand{\blankline}{\quad\pagebreak[2]}
%%%%%%%%%%%%%%%%%%%%%%%%%%%%%%%%%%%%%%%%%%%%%%%%%%%%%%%%%%%%%%

% Modify Course title, instructor name, semester here %%%%%%%%

\title{EC330: Urban Economics}
\author{Andrew Dickinson}
\date{Summer, 2021}

%%%%%%%%%%%%%%%%%%%%%%%%%%%%%%%%%%%%%%%%%%%%%%%%%%%%%%%%%%%%%%

% Don't touch this %%%%%%%%%%%%%%%%%%%%%%%%%%%%%%%%%%%%%%%%%%%
\usepackage[sc]{mathpazo}
\linespread{1.05} % Palatino needs more leading (space between lines)
\usepackage[T1]{fontenc}
\usepackage[mmddyyyy]{datetime}% http://ctan.org/pkg/datetime
\usepackage{advdate}% http://ctan.org/pkg/advdate
\newdateformat{syldate}{\twodigit{\THEMONTH}/\twodigit{\THEDAY}}
\newsavebox{\MONDAY}\savebox{\MONDAY}{Mon}% Mon
\newcommand{\week}[1]{%
	%  \cleardate{mydate}% Clear date
	% \newdate{mydate}{\the\day}{\the\month}{\the\year}% Store date
	\paragraph*{\kern-2ex\quad #1, \syldate{\today} - \AdvanceDate[4]\syldate{\today}:}% Set heading  \quad #1
	%  \setbox1=\hbox{\shortdayofweekname{\getdateday{mydate}}{\getdatemonth{mydate}}{\getdateyear{mydate}}}%
	\ifdim\wd1=\wd\MONDAY
	\AdvanceDate[7]
	\else
	\AdvanceDate[7]
	\fi%
}
\usepackage{setspace}
\usepackage{multicol}
%\usepackage{indentfirst}
\usepackage{fancyhdr,lastpage}
\usepackage{url}
\pagestyle{fancy}
\usepackage{hyperref}
\usepackage{lastpage}
\usepackage{amsmath}
\usepackage{layout}

\lhead{}
\chead{}
%%%%%%%%%%%%%%%%%%%%%%%%%%%%%%%%%%%%%%%%%%%%%%%%%%%%%%%%%%%%%%

% Modify header here %%%%%%%%%%%%%%%%%%%%%%%%%%%%%%%%%%%%%%%%%
\rhead{\footnotesize EC330: Syllabus}

%%%%%%%%%%%%%%%%%%%%%%%%%%%%%%%%%%%%%%%%%%%%%%%%%%%%%%%%%%%%%%
% Don't touch this %%%%%%%%%%%%%%%%%%%%%%%%%%%%%%%%%%%%%%%%%%%
\lfoot{}
\cfoot{\small \thepage/\pageref*{LastPage}}
\rfoot{}

\usepackage{array, xcolor}
\usepackage{color,hyperref}
\definecolor{clemsonorange}{HTML}{EA6A20}
\hypersetup{colorlinks,breaklinks,linkcolor=clemsonorange,urlcolor=clemsonorange,anchorcolor=clemsonorange,citecolor=black}

\begin{document}
	
	\maketitle
	
	\blankline
	
	\begin{tabular*}{.93\textwidth}{@{\extracolsep{\fill}}lr}
		
		%%%%%%%%%%%%%%%%%%%%%%%%%%%%%%%%%%%%%%%%%%%%%%%%%%%%%%%%%%%%%%
		
		% Modify information %%%%%%%%%%%%%%%%%%%%%%%%%%%%%%%%%%%%%%%%%
		E-mail: \texttt{adickin3@uoregon.edu}  \\
		
		Office Hours: \textbf{TBD}  &  Class Hours:  \textbf{M,T,W,R 12:00-13:50 PST} \\
		
		Office: \textbf{Zoom} & Class Room: \textbf{Zoom}\\

		&\\
		\hline
		\\
	\end{tabular*}
	
	\href{https://www.youtube.com/watch?v=aOIvB2YtAhY}{Please take time to read through this document}
	
	\vspace{5 mm}
	
	% First Section %%%%%%%%%%%%%%%%%%%%%%%%%%%%%%%%%%%%%%%%%%%%
	
	\section*{Course Description}
		
	Economics is all about people and incentives. Urban economics is about people, incentives, \textbf{and} their location choices. \\
	
	\noindent This course will probe a broad array of topics related to phenomena that arise from cities. We will begin by analyzing basic economic principles that describe the existence of cities. The first half of the course will introduce students to various models and theories commonly used in urban economics. Equipped with these tools,  we will cover intra- and intercity topics such as inequality, transportation, growth, and environmental issues. Students will be introduced to cornerstone models, theories, and practices in the field.
	
	

	
	
	
	
	% Second Section %%%%%%%%%%%%%%%%%%%%%%%%%%%%%%%%%%%%%%%%%%%
	\section*{Remote Learning}
	
Due to the lingering circumstances that we all have had to adjust to this year, this course will be delivered remotely. By remotely I do not mean asynchronously. This this course is not designed to be asynchronous and lectures will be delivered during regularly scheduled class times. However, lectures will be recorded and posted to the canvas page following the end of class. \textbf{Given the format of this summer course, attending and participating during the live lectures is strongly encouraged. This class will move very quickly. If anything attending live lectures will ensure that you are keeping up with all the material as it is being delivered.} A portion of the grade will incentivise live attendence and/or keeping up with the recordings on the same day. More information about that portion of the grade is discussed in further detail in the blank section.

The midterm and final will be held at the regularly scheduled course times. However the midterm may be subject to be changed to a timed exam during a 12 hour window on Friday, July 30th. This will depend on availability. If it changes the syllabus on the course github + canvas will be updated. Zoom links will be posted on canvas under the Zoom Meetings tab. Additionally, office hours will be held following the lectures from 2:00-2:30 M-Th or by appointment which you can make by sending me an email.
	
We all are still facing a lot of continuing changes and challenges due to the pandemic. I am going to do my best to offer you a high-quality, remotely instructed course. However, I imagine there will be hiccups along the way. I respectfully request your patience. I know you are also dealing with a lot of challenges so I offer my own patience to you. Let's make the best of this situation.

\section*{Academic Dishonesty}
 I want to emphasize the course and university policies on academic dishonesty. I am emphasizing these policies because I want to make it clear: I do not tolerate cheating.   Please see the following from the University for some specifics.
	
\begin{quote}
\textit{The University Student Conduct Code (available at \href{https://dos.uoregon.edu/conduct}{conduct.uoregon.edu}) defines academic misconduct. Students are prohibited from committing or attempting to commit any act that constitutes academic misconduct. By way of example, students should not give or receive (or attempt to give or receive) unauthorized help on assignments or examinations without explicit permission from the instructor. Students should properly acknowledge and document all sources of information (e.g. quotations, paraphrases, ideas).} \\

\textit{If there is any question about whether a particular activity constitutes academic misconduct, it is the student’s obligation to clarify the matter with the instructor before engaging in or attempting to engage in the activity. Please contact me with any questions you have about academic misconduct.	Additional information about maintaining your academic integrity is available at integrity.uoregon.edu. Information about plagiarism is available at researchguides.uoregon.edu/citing-plagiarism.} \\
	
	
	\hspace{3in }-- \textit{The Office of the Dean of Students}
\end{quote}

\section*{Course Setup}

	% Third Section %%%%%%%%%%%%%%%%%%%%%%%%%%%%%%%%%%%%%%%%%%%
	
	\section*{Prerequisites}
	You must have successfully completed \textbf{EC201} \textbf{prior} to taking EC330. Economic concepts are often presented in mathematical models. This course assumes proficiency in algebra and geometry. The models we will work with in this course will be graph based. Thus calculus is \textbf{not} required.
	
	% Fourth Section %%%%%%%%%%%%%%%%%%%%%%%%%%%%%%%%%%%%%%%%%%%
	
	\section*{Course Objectives}
	
	The goal of this course is for students to glean insight on models and topics related to urban economics. Successful students will be able to carefully articulate why cities exist, what drives differences in success across cities, and the efficacy of various place-based policies. Additionally, successful students will demonstrate a thorough understanding of the mathematical (algebraic and graphical) models developed in this course.
	
	% Fifth Section %%%%%%%%%%%%%%%%%%%%%%%%%%%%%%%%%%%%%%%%%%%
	\
	\subsection*{Course materials}
	
	\noindent There is no official \textit{textbook} for this course. However, you will be expected to read the entirety of Edward Glaeser’s \href{https://www.amazon.com/Triumph-City-Greatest-Invention-Healthier/dp/0143120549}{Triumph of the City} (\textbf{TotC}) (\href{https://www.abebooks.com/book-search/title/triumph-of-the-city/author/glaeser/}{alternative link}). Exam questions will be based primarily on my lecture notes and TotC. Some of this course will follow \textit{Urban Economics} by Arthur O'Sullivan, although the book is not required for this course. The Duck Store is not carrying it for this course (although it may be for EC432). I don’t recommend that you purchase the 8th edition of O’Sullivan’s Urban Economics at The Duck Store. (Note that the high prices there are the fault of the publisher McGraw Hill). Instead, if you’d like to have a version of the textbook as an additional resource purchase a used copy of the 7th edition (or older) on the internet. A copy of the 8th edition is on reserve at the Knight Library and pdfs are easy to find on via Google search.\\
	
	
	\noindent The content of TotC will cover a lot of course material. It is the expectation that you keep up with the required reading that will be posted in lecture. Some class time will be dedicated to discussion of the required reading. TotC is a wonderful book that is interesting and easy to read. It presents much of the material of this course in an organized and accessable way that I cannot outdo in lecture. Thus the content of this book will be the subject of many questions on problem sets and exams and 10\% of your grade will be based on a short book report that is due at the end of the term.
	
	\subsection*{Problem sets}

All assignments will be submitted via canvas. Specific instructions for submission will be posted on canvas. Late assignments will not be accepted. You are encouraged to work on problem sets with other students, but all answers written on your problem set must be in your own words and turned in separately. Group sizes must be limited to four (though I am flexible) and you all must clearly indicate your group mates at the top of your assignment. Simply copying someone else’s work is cheating. If such duplication is detected all problem sets involved will receive a zero, regardless of which student did the work and which did the copying. \\

The two problem sets are intended to help students prepare for the exams. They will be a good indicator of your understanding and I will encourage you to use them as a study guide to prepare for the two exams in this course. The first problem set will be assigned week 1 and the second week 3. They will both be due the Wednesday before class 
	
	\subsection*{Quizzes}

		In order to keep anyone from falling behind, an undetermined number of quizzes will be announced \textbf{during lecture} and posted in the slides. You may think of these as pop quizzes. Expect them to be \textbf{short and easy practice questions} that will help prepare you for the exams. They will not be announced before nor after being assigned and each quiz is due within 24 hours following the end of class. They will be graded quickly and solutions will be given in the following lecture. The lowest quiz score will be dropped. \\
		
		The quizzes are intended to encourage attendence to live lectures and/or expeditious watching of recorded lectures. This course is 10 weeks of lectures condensed into 4 weeks. It is imperitive that you stay on top of the lecture material or else you will fall behind extremely quickly. There is not much time to make up for a slow start in 4 weeks.

	\subsection*{Exams}
	
	There will be a midterm and a final exam. The final exam will be comprehensive, but it will be more heavily weighted toward material presented in the latter half of the course. No make-up exams will be given. Let me know as soon as possible if you cannot make it to the midterm. If a serious illness or emergency prevents you from taking an exam at the scheduled date, contact me \textbf{before the exam} via email (adickin3@uoregon.edu).\\
	
	\noindent Excused absences from the midterm will lead to a reweighting of the final exam to \textbf{65 \%} of your grade.  Excused absences will only be granted in extraordinary circumstances. An \textbf{unexcused} absence from the midterm will result in scoring a zero on that exam, with devastating effects on your course grade. An unexcused absence for the final exam will likely result in a failing grade.
	
	\noindent Make-up exams will not be given for any reason. In the case of a missed midterm due to unanticipated emergency situations, the student will be allowed to put the weight of the missed midterm on the final, provided notification is received as soon as possible and there is verification of the emergency. \textbf{DO NOT} take this class if you already know you cannot make one of the scheduled exams. The midterm will take place during scheduled class time on Thursday, July 29th and will cover all material prior to the midterm. The final will be cumulative and be held during class time during the final lecture.\\
	
	
	\subsection*{Book Report}
	
		As mentioned above, over the term you are expected to read Edward Glaeser’s Triumph of the City. You will be asked to write a report based on your reflections on the content and themes of the book. The specific assignment and a detailed rubric for this written report will be provided early in the term, and the report itself will be due on Canvas on \textbf{TBD}.

	\subsection*{Grading Policy}
	There will be a total of 500 points available. The following gives the breakdown of how points will be allocated 
	\begin{itemize}
		\item \underline{\textbf{35\%}}  Final exam (1x, 175 points)
		\item \underline{\textbf{30\%}}  Midterm exam (1x, 150 points)
		\item \underline{\textbf{15\%}}  Problem sets (2x, 37.5 points each)
		\item \underline{\textbf{10\%}}  Quizzes  (5?x, 50 points total)
		\item \underline{\textbf{10\%}}  Triumph of the Cities report (1x, 50 points)
		
	\end{itemize}

Due dates for assignments and exam dates can be found on the class schedule starting on page 6 of the syllabus. The economics department has a uniform (the same across all classes) grading standard.  The department takes grade inflation very seriously. In 300 and 400 level classes, roughly 65\% of the class will receive A's and B's. Your grade will be determined relative to your peers, so during the course, I will not be able to tell you what your exact letter grade is at any point in time, because it depends on the scores of everyone else at the end of the course.

Your grade will be determined relative to your peers, so during the course, I will not be able to tell you what your exact letter grade is at any point in time, because it depends on everyone’s overall scores of the class.

	% Add a figure %%%%%%%%%%%%%%%%%%%%%%%%%%%%%%%%%%%%%%%%%%%
	
	
	% Fifth Section %%%%%%%%%%%%%%%%%%%%%%%%%%%%%%%%%%%%%%%%%%%
	
	\section*{Course Policies}
	
	\subsection*{Emailing}
	
	I will try my best to be as available as possible via email. This is my preferred method of communication. Please read the syllabus to make sure you don't email me with questions that are already answered.
	
	\subsection*{Attendence}
	
	Like all synchronous courses, attendence to lecture is mandatory. I have written a program that takes automates taking attendence to the live lectures on zoom. I also know when students watch the recordings. While attendence does not directly affect your grade, it will influence letter grades that are close to the cutoff at the end of the course.
	
	\subsection*{During Lectures}
	
Please be respectful during the zoom lectures.  I will stop periodically to ask if anyone has questions. Feel free to use the ``raise-hand'' feature during a lecture to ask something. I reserve the right to kick people out of the lecture for any behavior I deem as inappropriate.  
	
	\subsection*{Copyright Notice}
	
The materials used in this class, including, but not limited to, lecture slides, exams, and problem sets are protected by copyright. You are granted permission to access materials and make copies and derivative works for your personal use. \textbf{You may not record the lectures}. You may not distribute course materials to any other person, regardless of student status, without express written permission from me. The sharing of class materials without the specific, express approval of the instructor may be a violation of the University’s Student Conduct Code (Section V.2.f.F), which could result in further disciplinary action. This includes, among other things, uploading class materials to websites for the purpose of sharing those materials with other current or future students. Additionally, any unauthorized copying of class materials is a violation of federal law and may result in further action.

	
	\subsection*{Accommodations for Disabilities}
	
	The University of Oregon is working to create inclusive learning environments. Please notify me if there are any aspects of this course that result in disability-related barriers to your participation. For more information or assistance, contact the Accessible Education Center: 164 Oregon Hall | 541-346-1155 | aec.uoregon.edu.\\
	
	If you require special accommodations of any kind due to a documented disability please have the Accessible Education Center send me a letter verifying your need and detailing the appropriate accommodations. This is typically done automatically by the AEC at the beginning of a term. If there are new accommodations during a term or any changes to any existing accommodations please keep me advised as soon as possible.\\
	
\noindent	If your accommodations involve any proctoring of exams at the AEC you will be responsible for scheduling those exams with sufficient anticipation with the AEC. Keep in mind that proctored midterms need to be scheduled at least seven days in advance and that proctored final exams need to be scheduled by the 5:00pm of the Friday of week 8 of the course. As you can see on the next page, the exams for this course are already set. I recommend that you make the necessary arrangements with AEC now.
	
	\subsection*{Diversity}
	
	The University of Oregon is dedicated to the principles of equal opportunity and freedom from unfair discrimination for all members of the university community and an acceptance of true diversity as an affirmation of individual identity within a welcoming community. This course is committed to upholding these principles by encouraging the exploration, engagement, and expression of distinct perspectives and diverse identities.
	All of us associated with the course—you included—are expected to value each class member’s experiences and contributions and to communicate disagreements respectfully. Please notify me if you feel aspects of the course undermine these principles in any way. You may also notify the Department of Economics at 541- 346-8845. For additional assistance and resources, you are also encouraged to contact the following campus services:
	\begin{itemize}
		\item Office of Equity and Inclusion: 1 Johnson Hall | 541-346-3175 | inclusion.uoregon.edu
		\item Center on Diversity and Community: 54 Susan Campbell Hall | 541-346-3212 | codac.uoregon.edu
		\item Bias Education and Response Team: 541-346-1134 | brt@uoregon.edu | dos.uoregon.edu/bias
		
	\end{itemize}
	% Course Schedule %%%%%%%%%%%%%%%%%%%%%%%%%%%%%%%%%%%%%%%%%%%
	
	\newpage
	\section*{Schedule}
	
	Below is a rough course outline that is subject to change. Especially the last few lectures.\\ (TotC = Triumph of the City)
	
	% Set first date of the semester (for some reason this is a week before what comes up, but that's easy to get around)
	\noindent\rule[0.5ex]{\linewidth}{1pt}
	\textbf{Week 01} \textit{Introduction + Review + City Size + Clustering}
	\begin{itemize}
		\item Lecture 01 (July 19): Welcome \& Introduction to Urban Economics
		\item Lecture 02 (July 20): EC 201 Review \& The 5 Axioms of Urban Economics
		\item Lecture 03 (July 21): City Size
		\item Lecture 04 (July 22):  Firm Clustering \& City Growth
		\item[]\underline{\textbf{Action Items}}
		\begin{itemize}
			\item Read intro + chapters 1, and 2 of \textit{ToTC}
		\end{itemize}
	\end{itemize}
	\noindent\rule[0.5ex]{\linewidth}{1pt}
	
	\textbf{Week 02} \textit{Cities \& Rents + Neighborhood choice}
	\begin{itemize}
		\item Lecture 05 (July 26): Rents and urban land use (part i)
		\item Lecture 06 (July 27): Rents and urban land use (part ii)
		\item Lecture 07 (July 28): Neighborhood choice 
		\item Lecture 08 (July 29):  \textbf{Midterm}*
		\item[]\underline{\textbf{Action Items}}
		\begin{itemize}
				\item Read chapters 3, 4, and 5 of \textit{ToTC}
				\item Problem set 01 due on July 28th by the start of class
		\end{itemize}
	\end{itemize}
	\noindent\rule[0.5ex]{\linewidth}{1pt}
	
		\textbf{Week 03} \textit{Cities \& Local labor markets + Place based policies}
	\begin{itemize}
		\item Lecture 09 (August 02): Local labor markets (part i)
		\item Lecture 10 (August 03): Local labor markets (part ii)
		\item Lecture 11 (August 04): Introduction to place-based policy
		\item Lecture 12 (August 05): Minimum wage
		\item[]\underline{\textbf{Action Items}}
		\begin{itemize}
				\item Read chapters 6 and 7 of \textit{ToTC}
		\end{itemize}
	\end{itemize}
	\noindent\rule[0.5ex]{\linewidth}{1pt}
	
		\textbf{Week 04} \textit{Cities \& Housing + Transit}
	\begin{itemize}
		\item Lecture 13 (August 09): Housing policy
		\item Lecture 14 (August 10): Automobiles
		\item Lecture 15 (August 11): Urban transit
		\item Lecture 16 (August 12):  \textbf{Final Exam}
		\item[]\underline{\textbf{Action Items}}
		\begin{itemize}
				\item Read chapters 8 and 9 of \textit{ToTC}
				\item ToTC report due on Canvas due midnight Sunday, August 15
		\end{itemize}
	\end{itemize}
	\noindent\rule[0.5ex]{\linewidth}{1pt}
	
\end{document}



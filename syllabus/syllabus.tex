% Don't touch this %%%%%%%%%%%%%%%%%%%%%%%%%%%%%%%%%%%%%%%%%%%
\documentclass[11pt]{article}
\usepackage{fullpage}
\usepackage[left=1in,top=1in,right=1in,bottom=1in,headheight=3ex,headsep=3ex]{geometry}
\usepackage{graphicx}
\usepackage{float}

\newcommand{\blankline}{\quad\pagebreak[2]}
%%%%%%%%%%%%%%%%%%%%%%%%%%%%%%%%%%%%%%%%%%%%%%%%%%%%%%%%%%%%%%

% Modify Course title, instructor name, semester here %%%%%%%%

\title{EC330: Urban Economics}
\author{Andrew Dickinson}
\date{Summer, 2021}

%%%%%%%%%%%%%%%%%%%%%%%%%%%%%%%%%%%%%%%%%%%%%%%%%%%%%%%%%%%%%%

% Don't touch this %%%%%%%%%%%%%%%%%%%%%%%%%%%%%%%%%%%%%%%%%%%
\usepackage[sc]{mathpazo}
\linespread{1.05} % Palatino needs more leading (space between lines)
\usepackage[T1]{fontenc}
\usepackage[mmddyyyy]{datetime}% http://ctan.org/pkg/datetime
\usepackage{advdate}% http://ctan.org/pkg/advdate
\newdateformat{syldate}{\twodigit{\THEMONTH}/\twodigit{\THEDAY}}
\newsavebox{\MONDAY}\savebox{\MONDAY}{Mon}% Mon
\newcommand{\week}[1]{%
	%  \cleardate{mydate}% Clear date
	% \newdate{mydate}{\the\day}{\the\month}{\the\year}% Store date
	\paragraph*{\kern-2ex\quad #1, \syldate{\today} - \AdvanceDate[4]\syldate{\today}:}% Set heading  \quad #1
	%  \setbox1=\hbox{\shortdayofweekname{\getdateday{mydate}}{\getdatemonth{mydate}}{\getdateyear{mydate}}}%
	\ifdim\wd1=\wd\MONDAY
	\AdvanceDate[7]
	\else
	\AdvanceDate[7]
	\fi%
}
\usepackage{setspace}
\usepackage{multicol}
%\usepackage{indentfirst}
\usepackage{fancyhdr,lastpage}
\usepackage{url}
\pagestyle{fancy}
\usepackage{hyperref}
\usepackage{lastpage}
\usepackage{amsmath}
\usepackage{layout}

\lhead{}
\chead{}
%%%%%%%%%%%%%%%%%%%%%%%%%%%%%%%%%%%%%%%%%%%%%%%%%%%%%%%%%%%%%%

% Modify header here %%%%%%%%%%%%%%%%%%%%%%%%%%%%%%%%%%%%%%%%%
\rhead{\footnotesize EC330: Syllabus}

%%%%%%%%%%%%%%%%%%%%%%%%%%%%%%%%%%%%%%%%%%%%%%%%%%%%%%%%%%%%%%
% Don't touch this %%%%%%%%%%%%%%%%%%%%%%%%%%%%%%%%%%%%%%%%%%%
\lfoot{}
\cfoot{\small \thepage/\pageref*{LastPage}}
\rfoot{}

\usepackage{array, xcolor}
\usepackage{color,hyperref}
\definecolor{clemsonorange}{HTML}{EA6A20}
\hypersetup{colorlinks,breaklinks,linkcolor=clemsonorange,urlcolor=clemsonorange,anchorcolor=clemsonorange,citecolor=black}

\begin{document}
	
	\maketitle
	
	\blankline
	
	\begin{tabular*}{.93\textwidth}{@{\extracolsep{\fill}}lr}
		
		%%%%%%%%%%%%%%%%%%%%%%%%%%%%%%%%%%%%%%%%%%%%%%%%%%%%%%%%%%%%%%
		
		% Modify information %%%%%%%%%%%%%%%%%%%%%%%%%%%%%%%%%%%%%%%%%
		E-mail: \texttt{adickin3@uoregon.edu}  \\
		
		Office Hours: \textbf{TBD}  &  Class Hours:  \textbf{M,T,W,R 12:00-13:45 PST} \\
		
		Office: \textbf{Zoom} & Class Room: \textbf{Zoom}\\

		&\\
		\hline
	\end{tabular*}
	
	\vspace{5 mm}
	
	% First Section %%%%%%%%%%%%%%%%%%%%%%%%%%%%%%%%%%%%%%%%%%%%
	
	\section*{Course Description}
	
		
	Economics is all about people and incentives. Urban economics is about people, incentives, \textit{and} their location choices. \\
	
	\noindent This course will probe a broad array of topics related to phenomena that arise from cities. We will begin by analyzing basic economic principles that describe the existence of cities. The first half of the course will introduce students to various models and theories commonly used in urban economics. Equipped with these tools,  we will cover inter and intra - city topics such as inequality, transportation, growth, and environmental issues. Students will be introduced to cornerstone models, theories, and practices in the field.
	
	

	
	
	
	
	% Second Section %%%%%%%%%%%%%%%%%%%%%%%%%%%%%%%%%%%%%%%%%%%
	\section*{Remote Learning}
	
This course is remote and thus live zoom\textbf{ lectures will be delivered during the regularly scheduled class time}. \textbf{Exams will also be held at the regularly scheduled course times.} Information for zoom meetings will be posted on canvas. \textbf{I will record the lectures} and post them immediately after. Additionally, I will set aside a few minutes at the end of each lecture for questions that will not be recorded. Attending the zoom lectures is not mandatory, but it is encouraged. Please make an effort to keep up with the material.
	
We all are facing a lot of continuing changes and challenges this quarter. I am going to do my best to offer you a high-quality, remotely instructed course. However, I imagine there will be hiccups along the way, and I respectfully request your patience. I know you are also dealing with a lot of challenges, so I offer my own patience to you. Let's make the best of this situation.



\section*{Academic Dishonesty}
 I want to emphasize the course and university policies on academic dishonesty. I am emphasizing these policies because I want to make it clear: I do not tolerate cheating.   Please see the following from the University for some specifics.
	
	
\begin{quote}

The University Student Conduct Code (available at conduct.uoregon.edu) defines academic misconduct. Students are prohibited from committing or attempting to commit any act that constitutes academic misconduct. By way of example, students should not give or receive (or attempt to give or receive) unauthorized help on assignments or examinations without explicit permission from the instructor. Students should properly acknowledge and document all sources of information (e.g. quotations, paraphrases, ideas).\\

 If there is any question about whether a particular activity constitutes academic misconduct, it is the student’s obligation to clarify the matter with the instructor before engaging in or attempting to engage in the activity. Please contact me with any questions you have about academic misconduct.
	Additional information about maintaining your academic integrity is available at integrity.uoregon.edu. Information about plagiarism is available at researchguides.uoregon.edu/citing-plagiarism.\\
	
	
	\hspace{3in }-- \textit{From the Office of the Dean of Students}

\end{quote}
\section*{Course Setup}
		
		

	
	% Third Section %%%%%%%%%%%%%%%%%%%%%%%%%%%%%%%%%%%%%%%%%%%
	
	\section*{Prerequisites/Corequisites}
	You must have successfully completed \textbf{EC201} \textbf{prior} to taking EC330. Economic concepts are often presented in mathematical models. This course assumes proficiency in algebra and geometry (i.e., working with graphs). Calculus is \textbf{not} required.
	
	% Fourth Section %%%%%%%%%%%%%%%%%%%%%%%%%%%%%%%%%%%%%%%%%%%
	
	
	
	\section*{Course Objectives}
	
	The goal of this course is for students to glean insight on models and topics related to urban economics. Successful students will be able to carefully articulate why cities exist, what drives differences in success across cities, and the efficacy of various place-based policies. Additionally, successful students will demonstrate a thorough understanding of the mathematical (algebraic and graphical) models developed in this course.
	
	% Fifth Section %%%%%%%%%%%%%%%%%%%%%%%%%%%%%%%%%%%%%%%%%%%
	\
	\section*{Course Books}
	
	
	\noindent There is no official \textit{textbook} for this course. However, you will be expected to read the entirety of Edward Glaeser’s \textit{Triumph of the City} (TotC). Exam questions will be based primarily on my lecture notes and TotC. Some of this course will follow \textit{Urban Economics} by Arthur O'Sullivan, although the book is not required for this course. The Duck Store is not carrying it for this course (although it may be for EC 432). I don’t recommend that you purchase the 8th edition of O’Sullivan’s Urban Economics at The Duck Store. (Note that the high prices there are the fault of the publisher McGraw Hill). Instead, if you’d like to have a version of the textbook as an additional resource purchase a used copy of the 7th edition (or older) on the internet. A copy of the 8th edition is on reserve at the Knight Library.\\
	
	
	\noindent The content of TotC will be the subject of questions on homework and exams, and you will be asked to turn in a report on the book at the end of the term. 
	

	

	
	
	\subsection*{Intro Quiz}

		There will be a canvas EC201 review quiz that opens after the second lecture. This quiz is to incentivize you to review some of the key concepts you learned in EC201 that will be crucial for this course. We will go over what will be on the quiz during the second lecture. 
		

	\subsection*{Exams}
	
	There will be a midterm and a final exam. The final exam will be comprehensive, but it will be more heavily weighted toward material presented in the latter half of the course. No make-up exams will be given. Let me know as soon as possible if you cannot make it to the midterm. If a serious illness or emergency prevents you from taking an exam at the scheduled date, contact me \textbf{before the exam} at jmorehou@uoregon.edu.\\
	
	\noindent If you have an \textbf{excused} absence from the midterm you may earn the privilege of having the final exam reweighted to \textbf{65 \%} of your grade.  Excused absences will only be granted in extraordinary circumstances.\\
	
	\noindent An \textbf{unexcused} absence from the midterm will result in scoring a zero on that exam, with devastating effects on your course grade. An unexcused absence from the final exam will likely result in a failing grade for the course.
	
	
	\subsection*{Book Report}
	
		As mentioned above, over the term you are expected to read Edward Glaeser’s Triumph of the City. You will be asked to write a report based on your reflections on the content and themes of the book. The specific assignment and a detailed rubric for this written report will be provided early in the term, and the report itself will be due on Canvas on \textbf{Sunday, May 30th}.

	

	\subsection*{Grading Policy}
	There will be a total of 500 points available. The following gives the breakdown of how points will be allocated 
	\begin{itemize}
		\item \underline{\textbf{35\%}}  Final exam (1x, 175 points)
		\item \underline{\textbf{30\%}} Midterm exam (1x, 150 points)
		\item \underline{\textbf{20\%}}   Homeworks (4x, 25 points each)
		\item \underline{\textbf{10\%}} \textit{Triumph of the Cities} report (1x, 50 points)
		\item \underline{\textbf{4\%}}  Intro Quiz  (1x, 20 points)
		\item \underline{\textbf{1\%}} Letter of Introduction (1x, 5 points)
		
	\end{itemize}

Due dates for assignments and exam dates can be found on the class schedule starting on page 6 of the syllabus. The economics department has a uniform (the same across all classes) grading standard.  The department takes grade inflation very seriously. In 300 and 400 level classes, roughly 65\% of the class will receive A's and B's. Your grade will be determined relative to your peers, so during the course, I will not be able to tell you what your exact letter grade is at any point in time, because it depends on the scores of everyone else at the end of the course.



	% Add a figure %%%%%%%%%%%%%%%%%%%%%%%%%%%%%%%%%%%%%%%%%%%
	
	
	% Fifth Section %%%%%%%%%%%%%%%%%%%%%%%%%%%%%%%%%%%%%%%%%%%
	
	
	\section*{Course Policies}
	
	
	
	\subsection*{During Lectures}
	
Please be respectful during the zoom lectures.  I will stop periodically to ask if anyone has questions. Feel free to use the ``raise-hand'' feature during a lecture to ask something. I reserve the right to kick people out of the lecture for any behavior I deem as inappropriate.  
	

	
	
	\subsection*{Homeworks and Exams}

All assignments will be submitted via canvas. Specific instructions for submission will be posted on canvas. You are allowed one late problem set in this class (note: this does \textbf{not} include the book-report). The late problem set \textbf{must} be cleared with me prior to the due date and must be turned in \textbf{before of the next class period.} Make-up exams will not be given for any reason. In the case of a missed midterm due to unanticipated emergency situations, the student will be allowed to put the weight of the missed midterm on the final, provided notification is received as soon as possible and there is verification of the emergency. \textbf{DO NOT} take this class if you already know you cannot make one of the scheduled exams. The midterm will occur during week 6 and will cover all material prior to the midterm. The final will be cumulative.\\

	
	
	
	
	\noindent You are encouraged to work on problem sets with other students, but all answers written on your problem set must be in your own words. Simply copying someone else’s work is cheating. If such duplication is detected all problem sets involved will receive a zero, regardless of which student did the work and which did the copying.

	
	\subsection*{Copyright Notice}
	
The materials used in this class, including, but not limited to, lecture slides, exams, and problem sets are protected by copyright. You are granted permission to access materials and make copies and derivative works for your personal use. \textbf{You may not record the lectures}. You may not distribute course materials to any other person, regardless of student status, without express written permission from me. The sharing of class materials without the specific, express approval of the instructor may be a violation of the University’s Student Conduct Code (Section V.2.f.F), which could result in further disciplinary action. This includes, among other things, uploading class materials to websites for the purpose of sharing those materials with other current or future students. Additionally, any unauthorized copying of class materials is a violation of federal law and may result in further action.

	
	\subsection*{Accommodations for Disabilities}
	The University of Oregon is working to create inclusive learning environments. Please notify me if there are any aspects of this course that result in disability-related barriers to your participation. For more information or assistance, contact the Accessible Education Center: 164 Oregon Hall | 541-346-1155 | aec.uoregon.edu.\\
	
	If you require special accommodations of any kind due to a documented disability please have the Accessible Education Center send me a letter verifying your need and detailing the appropriate accommodations. This is typically done automatically by the AEC at the beginning of a term. If there are new accommodations during a term or any changes to any existing accommodations please keep me advised as soon as possible.\\
	
\noindent	If your accommodations involve any proctoring of exams at the AEC you will be responsible for scheduling those exams with sufficient anticipation with the AEC. Keep in mind that proctored midterms need to be scheduled at least seven days in advance and that proctored final exams need to be scheduled by the 5:00pm of the Friday of week 8 of the course. As you can see on the next page, the exams for this course are already set. I recommend that you make the necessary arrangements with AEC now.
	
	\subsection*{Diversity}
	
	The University of Oregon is dedicated to the principles of equal opportunity and freedom from unfair discrimination for all members of the university community and an acceptance of true diversity as an affirmation of individual identity within a welcoming community. This course is committed to upholding these principles by encouraging the exploration, engagement, and expression of distinct perspectives and diverse identities.
	All of us associated with the course—you included—are expected to value each class member’s experiences and contributions and to communicate disagreements respectfully. Please notify me if you feel aspects of the course undermine these principles in any way. You may also notify the Department of Economics at 541- 346-8845. For additional assistance and resources, you are also encouraged to contact the following campus services:
	\begin{itemize}
		\item Office of Equity and Inclusion: 1 Johnson Hall | 541-346-3175 | inclusion.uoregon.edu
		\item Center on Diversity and Community: 54 Susan Campbell Hall | 541-346-3212 | codac.uoregon.edu
		\item Bias Education and Response Team: 541-346-1134 | brt@uoregon.edu | dos.uoregon.edu/bias
		
	\end{itemize}
	% Course Schedule %%%%%%%%%%%%%%%%%%%%%%%%%%%%%%%%%%%%%%%%%%%
	
	\newpage
	\section*{Schedule}
	
	Below is a rough course outline. Remember: TotC = Triumph of the City. Some lecture topics are subject to change. You are, of course, free to read TotC at your own pace. 
	
	% Set first date of the semester (for some reason this is a week before what comes up, but that's easy to get around)
	\noindent\rule[0.5ex]{\linewidth}{1pt}
	\textbf{Week 1} \textit{Introduction + Review}
	\begin{itemize}
		\item Lecture 1 (March 29th): Welcome \& Introduction to Urban Economics
		\item Lecture 2 (March 31st): EC 201 Review \& The 5 Axioms of Urban Economics
		\item[]\underline{\textbf{Action Items}}
		\begin{itemize}
			\item Review EC201
			\item Read Intro \& Chapter 1 of \textit{ToTC}
			\item \textbf{Intro Quiz}: EC201 and Algebra Review. Opens Mar 31, closes April 4th @ 11:59 PM
			\item Letter of Introduction due by April 4th @ 11:59 PM
		\end{itemize}
	\end{itemize}
	\noindent\rule[0.5ex]{\linewidth}{1pt}
	\textbf{Week 2} \textit{Cities \& Firm Clustering}
	\begin{itemize}
		\item Lecture 3 (April 5th): City Size
		\item Lecture 4 (April 7th):  Firm Clustering \& City Growth
		\item[]\underline{\textbf{Action Items}}
		\begin{itemize}
				\item Read Chapter 2 \& 3 of \textit{ToTC}
		\end{itemize}
	\end{itemize}
	\noindent\rule[0.5ex]{\linewidth}{1pt}
	
	
	
	\textbf{Week 3} \textit{Land Use \& Rents}
	
	\begin{itemize}
		\item Lecture 5 (April 12th):  Rents \& Urban Land Use
		\item Lecture 6 (April 14th): Rents \& Urban Land Use Part II
		\item[]\underline{\textbf{Action Items}}
		\begin{itemize}
			\item Read Chapter 4 of \textit{ToTC}
			\item HW I due on April 11th
	
		\end{itemize}
	\end{itemize}
	\noindent\rule[0.5ex]{\linewidth}{1pt}
	
	
	\textbf{Week 4} \textit{Neighborhood Choice}
	
	\begin{itemize}
		\item Lecture 7 (April 19th): Neighborhood Choice Part I
		\item Lecture 8 (April 21st):Neighborhood Choice Part II
		\item[]\underline{\textbf{Action Items}}         
		\begin{itemize}                                  
			\item Read Chapter 5 \& 6 of \textit{ToTC}                 
		\end{itemize}
	\end{itemize}
	\noindent\rule[0.5ex]{\linewidth}{1pt}
	
	
	\textbf{Week 5} \textit{Local Labor Markets}
	
	\begin{itemize}
		\item  Lecture 9 (April 26th): Local Labor Markets Part I
		\item  Lecture 10 (April 28th):  Local Labor Markets Part II
		\item[]\underline{\textbf{Action Items}}
		\begin{itemize}
			\item Read Chapter 7 of \textit{ToTC}
			\item HW II due on \textbf{April 30th}: note, this is a Friday, not Sunday
			
		\end{itemize}
	\end{itemize}
	\noindent\rule[0.5ex]{\linewidth}{1pt}
	
	
	\textbf{Week 6} \textit{\textbf{Midterm}}
	\begin{itemize}
		\item Lecture 11 (May 3rd): \textbf{Midterm}
		\item Lecture 12 (May 5th): Introduction to Place-Based Policy
		\item[]\underline{\textbf{Action Items}}
		\begin{itemize}
			\item \textbf{Midterm Exam}
			\item Read Chapter 8 of \textit{ToTC}
		\end{itemize}
	\end{itemize}
	\noindent\rule[0.5ex]{\linewidth}{1pt}
	
	
	
	\textbf{Week 7} \textit{Place-Based Policies}
	\begin{itemize}
		\item Lecture 13 (May 10th): Minimum Wage
		\item Lecture 14 (May 12th): Housing Policy
		\item[]\underline{\textbf{Action Items}}
		\begin{itemize}
			\item Read Chapter 9 of \textit{ToTC}
		\end{itemize}
	\end{itemize}
	\noindent\rule[0.5ex]{\linewidth}{1pt}
	
	
	\textbf{Week 8} \textit{Automobiles}
	\begin{itemize}
		\item Lecture 15 (May 17th): Automobiles Part I
		\item Lecture 16 (May 19th): Automobiles Part II
		\item[]\underline{\textbf{Action Items}}
		\begin{itemize}
			\item Finish \textit{ToTC}
			\item HW III due on May 23rd
		\end{itemize}
	\end{itemize}
	\noindent\rule[0.5ex]{\linewidth}{1pt}
	
	
	\newpage
	\textbf{Week 9} \textit{Transit and the Environment}
	
	\begin{itemize}
		\item Lecture 17 (May 24th): Urban transit 
		\item Lecture 18 (May 26th): Household sorting and the environment
		\item[]\underline{\textbf{Action Items}}
		\begin{itemize}
			\item \textbf{ToTC report due on Canvas by midnight on Sunday, May 30th}
		\end{itemize}
	\end{itemize}
	\noindent\rule[0.5ex]{\linewidth}{1pt}
	
	
	
	\textbf{Week 10} \textit{Topics in Urban Economics}
	\begin{itemize}
		\item  Lecture 19 (May 31st): The Geography of Income Inequality
		\item  Lecture 20 (June 2nd): TBD 
		\item[]\underline{\textbf{Action Items}}
		\begin{itemize}
			\item Study for the final
				\item HW IV due on canvas on June 6th
		\end{itemize}
	\end{itemize}
	\noindent\rule[0.5ex]{\linewidth}{1pt}
	
	
	
	\textbf{Week 11} \textit{\textbf{Week of Final}}

	\begin{itemize}
		\item \textbf{Final Exam June 8th @ 10:15 am PST}
	\end{itemize}
	
	
	
\end{document}


